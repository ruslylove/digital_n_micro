\documentclass{article}
\usepackage[utf8]{inputenc}
\usepackage{xcolor}
\usepackage{graphicx}
\usepackage{float}
\usepackage{minted}
\usepackage{circuitikz}
\usepackage{tikz}
\usetikzlibrary{shapes, arrows, calc, positioning, circuits.logic.US, circuits.logic.IEC}
\usepackage{geometry}
\geometry{a4paper, margin=1in}
\usepackage{hyperref}
\usemintedstyle{trac}

\usepackage{fancyhdr}
\pagestyle{fancy}
\fancyhf{}
\renewcommand{\headrulewidth}{0pt}
\fancyfoot[C]{\small 010153101 Digital Circuit and Microprocessor Fundamental \\ Semester 2/2025}
\fancyfoot[R]{\thepage}

% Redefine plain style to match fancy style (for the title page)
\fancypagestyle{plain}{%
  \fancyhf{}%
  \fancyfoot[C]{\small 010153101 Digital Circuit and Microprocessor Fundamental \\ Semester 2/2025}%
  \fancyfoot[R]{\thepage}%
  \renewcommand{\headrulewidth}{0pt}%
}

\title{Laboratory Exercise 5.5 \\ Combinational Circuit Design (Pre-session Lab)}
\author{}
\date{}

\begin{document}

\maketitle

\section*{Introduction}
This pre-session activity is designed to practice implementing combinational logic circuits using standard TTL (Transistor-Transistor Logic) Integrated Circuits (ICs). Students will design and build a simple 2-to-1 Multiplexer (MUX) and verify its operation using LEDs, a voltmeter, and an oscilloscope.

\section*{Objectives}
\begin{enumerate}
    \item To understand the operation of a 2-to-1 Multiplexer.
    \item To implement a combinational logic circuit using basic logic gates (AND, OR, NOT) on a breadboard.
    \item To verify circuit functionality using LEDs.
    \item To measure logic voltage levels using a digital multimeter.
    \item To observe signal transitions and propagation delay using an oscilloscope.
\end{enumerate}

\section*{Equipment Required}
\begin{itemize}
    \item DC Power Supply (+5V)
    \item Digital Multimeter
    \item Digital Oscilloscope

    \item Breadboard and Jumper Wires
    \item TTL ICs:
    \begin{itemize}
        \item 74LS04 (Hex Inverter / NOT Gate)
        \item 74LS08 (Quad 2-Input AND Gate)
        \item 74LS32 (Quad 2-Input OR Gate)
    \end{itemize}
    \item Resistors: $330\Omega$ (for LEDs), $10k\Omega$ (for pull-up/pull-down)
    \item DIP switches
    \item LEDs (Red/Green)
\end{itemize}

\section*{Part I: Theory and Logic Design}

A 2-to-1 Multiplexer (MUX) has two data inputs ($I_0, I_1$), one select input ($S$), and one output ($Y$). The operation is defined as follows:
\begin{itemize}
    \item If $S = 0$, the output $Y$ follows input $I_0$.
    \item If $S = 1$, the output $Y$ follows input $I_1$.
\end{itemize}

\subsection*{Truth Table}
Complete the truth table below to verify the logic.

\begin{table}[H]
    \centering
    \begin{tabular}{|c|c|c|c|}
        \hline
        \textbf{Select (S)} & \textbf{Input ($I_1$)} & \textbf{Input ($I_0$)} & \textbf{Output (Y)} \\
        \hline
        0 & x & 0 & 0 \\
        0 & x & 1 & 1 \\
        1 & 0 & x & 0 \\
        1 & 1 & x & 1 \\
        \hline
    \end{tabular}
    \caption{Function Table of 2-to-1 MUX}
    \label{tab:mux_truth}
\end{table}

The Boolean expression for the output $Y$ is:
\[ Y = \bar{S} \cdot I_0 + S \cdot I_1 \]

\subsection*{Logic Circuit Diagram}
Figure \ref{fig:mux_logic} shows the gate-level implementation of the 2-to-1 MUX.

\begin{figure}[H]
    \centering
    \begin{circuitikz}[circuitikz/logic ports=ieee]
        \draw
        (0,2) node[and port] (and1) {}
        (0,0) node[and port] (and2) {}
        (2.5,1) node[or port] (or1) {};
        
        % Align NOT gate output with AND1 input 2
        \draw (and1.in 2) -- ++(-0.5,0) node[not port, anchor=out, scale=0.5] (not1) {};
        
        % Align I0 with AND1 input 1
        \draw (and1.in 1) -- ++(-2.5,0) node[label=left:$I_0$] (I0) {};
        
        % Align I1 with AND2 input 2
        \draw (and2.in 2) -- ++(-2.5,0) node[label=left:$I_1$] (I1) {};
        
        % S Junction and connections
        \draw (not1.in) -- ++(-0.5,0) coordinate (s_split) node[circ] {};
        \draw (s_split) -- (s_split -| I0) node[label=left:$S$] (S) {};
        \draw (s_split) |- (and2.in 1);
        
        % AND outputs to OR inputs
        \draw (and1.out) -| (or1.in 1);
        \draw (and2.out) -| (or1.in 2);
        
        % Output Y
        \draw (or1.out) -- (3.5,1) node[right] {$Y$};
    \end{circuitikz}
    \caption{Logic Diagram of 2-to-1 MUX using Basic Gates.}
    \label{fig:mux_logic}
\end{figure}

Create a Quartus project for the 2-to-1 MUX circuit as follows:
\begin{enumerate}
    \item Create a new Quartus project for your DE0-CV board.
    \item \textbf{Implement the circuit using Schematic Design.} Draw the circuit using the Block Editor and appropriate logic gates (NOT, AND2, OR2).
    \item Compile the schematic design and perform a functional simulation to verify its correctness.
    \item Simulate the behavior by creating a vector waveform file (*.vwf). Specify the inputs ($S, I_0, I_1$) and output ($Y$).
    \item Run the simulation. The resulting waveforms should demonstrate the MUX behavior:
    \begin{itemize}
        \item When $S=0$, $Y$ should match $I_0$.
        \item When $S=1$, $Y$ should match $I_1$.
    \end{itemize}
\end{enumerate}



\section*{Part II: Hardware Implementation}

\subsection*{Schematic Diagram}
Construct the switch input circuits as shown in Figure \ref{fig:switch_circuit}. Then, connect the Multiplexer circuit on the breadboard as shown in Figure \ref{fig:schematic}. 
\begin{itemize}
    \item Use DIP switches for inputs $S, I_0, I_1$.
    \item Use $10k\Omega$ pull-down resistors to ensure logic '0' when OFF (Open).
    \item Use an LED with a $330\Omega$ series resistor to display the output $Y$.
\end{itemize}

\begin{figure}[H]
    \centering
    \begin{circuitikz}
        \draw (0,2) node[vcc] (vcc) {+5V}
        to[spst, l=DIP Switch] (0,0.5) -- (0,0)
        to[R, l_=$10k\Omega$] (0,-2) node[ground] (gnd) {};
        \draw (0,0) to[short,*-] (3.5,0) node[right] {To Input ($S, I_0, I_1$)};
        \node[right] at (0.5, 0.5) {Logic 1 (ON)};
        \node[right] at (0.5, -1) {Logic 0 (OFF)};
    \end{circuitikz}
    \caption{Input Switch Circuit (Active High).}
    \label{fig:switch_circuit}
\end{figure}

\begin{figure}[H]
    \centering
    \begin{circuitikz}[scale=0.9, transform shape]
        % Draw IC Blocks
        \draw (0,0) rectangle (2,4) node[midway, align=center] {7404\\(NOT)};
        \draw (4,0) rectangle (6,4) node[midway, align=center] {7408\\(AND)};
        \draw (8,0) rectangle (10,4) node[midway, align=center] {7432\\(OR)};
        
        % Pin Labels INSIDE blocks
        % 7404 Pins
        \node[right] at (0, 3) {\small 1};
        \node[left] at (2, 3) {\small 2};
        
        % 7408 Pins
        \node[right] at (4, 3.5) {\small 1};
        \node[right] at (4, 3) {\small 2};
        \node[left] at (6, 3.25) {\small 3}; % Out 1
        
        \node[right] at (4, 1.5) {\small 4};
        \node[right] at (4, 1) {\small 5};
        \node[left] at (6, 1.25) {\small 6}; % Out 2
        
        % 7432 Pins
        \node[right] at (8, 2.5) {\small 1};
        \node[right] at (8, 2) {\small 2};
        \node[left] at (10, 2.25) {\small 3}; % Out
        
        % Inputs
        \node (S) at (-2, 3) {$S$};
        \node (I0) at (-2, 3.5) {$I_0$};
        \node (I1) at (-2, 1) {$I_1$};
        
        % Wiring
        % S -> 7404 Pin 1 (Orange)
        \draw[orange, thick] (S) -- (0, 3);
        
        % 7404 Pin 2 -> 7408 Pin 2 (Input B of Gate 1) (Blue)
        \draw[blue, thick] (2, 3) -- (4, 3);
        
        % I0 -> 7408 Pin 1 (Input A of Gate 1) - Route OVER 7404 (Blue)
        \draw[blue, thick] (I0) -- (-0.5, 3.5) -- (-0.5, 4.5) -- (3.5, 4.5) -- (3.5, 3.5) -- (4, 3.5);
        
        % S -> 7408 Pin 4 (Input A of Gate 2) - Branching off S - Route UNDER 7404 (Blue)
        \draw[blue, thick] (-0.5, 3) node[circ, color=black]{} -- (-0.5, -0.5) -- (2.5, -0.5) -- (2.5, 1.5) -- (4, 1.5);
        
        % I1 -> 7408 Pin 5 (Input B of Gate 2) - Route UNDER 7404 (Blue)
        \draw[blue, thick] (I1) -- (-1, 1) -- (-1, -1) -- (3.2, -1) -- (3.2, 1) -- (4, 1);
        
        % 7408 Pin 3 -> 7432 Pin 1 (Green)
        \draw[green!60!black, thick] (6, 3.25) -- (7, 3.25) -- (7, 2.5) -- (8, 2.5);
        
        % 7408 Pin 6 -> 7432 Pin 2 (Green)
        \draw[green!60!black, thick] (6, 1.25) -- (7, 1.25) -- (7, 2) -- (8, 2);
        
        % Output
        % 7432 Pin 3 -> LED (Purple)
        \draw[violet, thick] (10, 2.25) -- (11, 2.25) to[leDo] (13, 2.25) to[R, l=$330\Omega$] (15, 2.25) node[ground] {};
        
        % Power connections text
        \node[align=center] at (5, -2.5) {Remember to connect Pin 14 to $V_{CC}$ and Pin 7 to GND for all ICs.};
        
    \end{circuitikz}
    \caption{Wiring Logic Diagram showing IC connections.}
    \label{fig:schematic}
\end{figure}

\begin{figure}[H]
    \centering
    \includegraphics[width=0.9\textwidth]{../public/mux_2_to_1_wiring_cropped.jpeg}
    \caption{Breadboard implementation of the 2-to-1 MUX.}
    \label{fig:breadboard}
\end{figure}

\textbf{Step-by-step Implementation:}
Refer to Figure \ref{fig:breadboard} for a visual guide of the breadboard layout.
\begin{enumerate}
    \item Place the 7404, 7408, and 7432 ICs on the breadboard.
    \item Connect Pin 14 ($V_{CC}$) to +5V and Pin 7 (GND) to Ground for all ICs.
    \item Construct the circuit according to Figure \ref{fig:mux_logic} using jumper wires.
    \item Connect the inputs $S, I_0, I_1$ to the switches/buttons.
    \item Connect the output $Y$ to the LED circuit.
\end{enumerate}

\section*{Part III: Measurement and Verification}

\subsection*{Functional Test}
Verify the circuit operation by manually checking all combinations of $S, I_0, I_1$.

\begin{enumerate}
    \item Set DIP switches keys to set $S, I_0, I_1$.
    \item Observe the LED status.
    \item Does it match the Truth Table in Table \ref{tab:mux_truth}?
\end{enumerate}

\subsection*{Voltage Measurement}
Use a Digital Multimeter (Voltmeter mode) to measure the output voltage levels.
\begin{enumerate}
    \item Connect the black probe to Ground (GND).
    \item Connect the red probe to the Output ($Y$).
    \item Measure the voltage when $Y$ is Logic 0. Record the value: \underline{\hspace{2cm}} V.
    \item Measure the voltage when $Y$ is Logic 1. Record the value: \underline{\hspace{2cm}} V.
    \item Do these values correspond to valid TTL logic levels?
\end{enumerate}

\subsection*{Oscilloscope Observation}
To visualize the switching behavior and signal integrity:
\begin{enumerate}
    \item Set $I_0$ to Logic 0 and $I_1$ to Logic 1.
    \item Connect Channel 1 of the oscilloscope to Select ($S$).
    \item Connect Channel 2 of the oscilloscope to Output ($Y$).
    \item Toggle $S$ (switch ON/OFF) and observe the transition on the oscilloscope.
    \item Capture the waveform showing the relationship between $S$ and $Y$.
    \item Set $I_0$ to Logic 1, $I_1$ to Logic 0, and repeat the observation.
\end{enumerate}



\vspace{2cm}
\noindent \textbf{Updated By:} R. Sutthaweekul \\
\textbf{Release Date:} 2026-01-17

\end{document}

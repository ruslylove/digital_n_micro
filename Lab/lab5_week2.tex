\documentclass{article}
\usepackage[utf8]{inputenc}
\usepackage{xcolor}
\usepackage{graphicx}
\usepackage{float}
\usepackage{minted}
\usepackage{circuitikz}
\usepackage{tikz}
\usetikzlibrary{shapes, arrows, calc, positioning, circuits.logic.US, circuits.logic.IEC}
\usepackage{geometry}
\geometry{a4paper, margin=1in}
\usepackage{hyperref}
\usemintedstyle{trac}

\usepackage{fancyhdr}
\pagestyle{fancy}
\fancyhf{}
\renewcommand{\headrulewidth}{0pt}
\fancyfoot[C]{\small 010153101 Digital Circuit and Microprocessor Fundamental \\ Semester 2/2025}
\fancyfoot[R]{\thepage}

% Redefine plain style to match fancy style (for the title page)
\fancypagestyle{plain}{%
  \fancyhf{}%
  \fancyfoot[C]{\small 010153101 Digital Circuit and Microprocessor Fundamental \\ Semester 2/2025}%
  \fancyfoot[R]{\thepage}%
  \renewcommand{\headrulewidth}{0pt}%
}

\title{Laboratory Exercise 5 - Week 2 \\ Registers and Applications}
\date{}

\begin{document}

\maketitle

The purpose of this exercise is to investigate latches, flip-flops, and registers. This second week focuses on advanced storage elements and their applications.

\setcounter{figure}{6}

\section*{Part IV}

Figure 7 shows a circuit with three different storage elements: a gated D latch, a positive-edge triggered D flip-flop, and a negative-edge triggered D flip-flop.

\begin{figure}[H]
    \centering
    % Placeholder for Figure 7: Three storage elements
    \begin{circuitikz}
    \ctikzset{logic ports=ieee}
    \ctikzset{flipflops/scale=1}

    % Define Styles based on Figure 6
    \tikzset{latch-style/.style={flipflop, flipflop def={t1=D, t3=E, t6=Q}}}
    \tikzset{ff-style/.style={flipflop, flipflop def={t1=D, t6=Q}}}

    % Nodes
    % 1. Gated D Latch
    \node[latch-style] (latch) at (0, 0) {};
    \node[above] at (latch.north) {Gated D Latch};

    % 2. Pos-Edge D FF
    \node[ff-style] (pos_ff) at (0, -3.5) {};
    \node[above] at (pos_ff.north) {Pos-Edge D FF};
    % Clock Wedge
    \draw (pos_ff.pin 3) ++(0, 0.2) -- ++(0.3, -0.2) -- ++(-0.3, -0.2);

    % 3. Neg-Edge D FF
    \node[ff-style] (neg_ff) at (0, -7) {};
    \node[above] at (neg_ff.north) {Neg-Edge D FF};
    % Clock Wedge
    \draw (neg_ff.pin 3) ++(0, 0.2) -- ++(0.3, -0.2) -- ++(-0.3, -0.2);
    % Bubble for Neg-Edge (Input side)
    \node[not port, anchor=out, scale=0.5] (clk_inv) at (neg_ff.pin 3) {};

    % Common Inputs
    \coordinate (D_in) at (-3.5, 1);
    \coordinate (Clk_in) at (-3.5, -2); % Positioned nicely

    \draw (D_in) node[left] {D} -- (D_in -| latch.pin 1) coordinate (D_top) -- (latch.pin 1);
    \draw (D_top) -- (D_top |- pos_ff.pin 1) -- (pos_ff.pin 1);
    \draw (D_top |- pos_ff.pin 1) -- (D_top |- neg_ff.pin 1) -- (neg_ff.pin 1);

    % Clock Wiring
    \draw (Clk_in) node[left] {Clock} -- (Clk_in -| latch.pin 3) coordinate (Clk_bus);
    
    % To Latch E
    \draw (Clk_bus) |- (latch.pin 3);
    
    % To Pos FF C
    \draw (Clk_bus) |- (pos_ff.pin 3);
    
    % To Neg FF C
    \draw (Clk_bus) |- (clk_inv.in);

    % Outputs
    \draw (latch.pin 6) -- ++(1, 0) node[right] {$Q_a$};
    \draw (pos_ff.pin 6) -- ++(1, 0) node[right] {$Q_b$};
    \draw (neg_ff.pin 6) -- ++(1, 0) node[right] {$Q_c$};

\end{circuitikz}

    \caption{Circuit for Part IV.}
    \label{fig:three_storage}
\end{figure}

\begin{figure}[H]
    \centering
    \input{tex/three_storage_elements_timing.tex}
    \caption{Waveforms for Part IV.}
    \label{fig:three_storage_timing}
\end{figure}

Implement and simulate this circuit using the Quartus software as follows:
\begin{enumerate}
    \item Create a new Quartus project.
    \item \textbf{First, implement the circuit using Schematic Design.} Draw the circuit using the Block Editor.
    \item Compile the schematic design and perform a functional simulation to verify its correctness.
    \item \textbf{Next, implement the circuit using Structural VHDL Design.}
    \item Write a VHDL file that instantiates the three storage elements. For this part you should no longer use the KEEP directive.
    \item Compile your code and use the Technology Map Viewer to examine the implemented circuit. Verify that the latch uses one lookup table and that the flip-flops are implemented using the flip-flops provided in the target FPGA.
    \item Create a Vector Waveform File (.vwf) that specifies the inputs and outputs of the circuit. Draw the inputs D and Clock as indicated in Figure 7. Use functional simulation to obtain the three output signals. Observe the different behavior of the three storage elements.
\end{enumerate}

\section*{Part V}

We wish to display the hexadecimal value of an 8-bit number A on the two 7-segment displays HEX3-2. We also wish to display the hex value of an 8-bit number B on the two 7-segment displays HEX1-0. The values of A and B are inputs to the circuit which are provided by means of switches SW7-0. To input the values of A and B, first set the switches to the desired value of A, store these switch values in a register, and then change the switches to the desired value of B. Finally, use an adder to generate the arithmetic sum S = A + B, and display this sum on the 7-segment displays HEX5-4. Show the carry-out produced by the adder on LEDR(0).

\begin{enumerate}
    \item Create a new Quartus project which will be used to implement the desired circuit on your DE-series board.
    \item \textbf{First, implement the circuit using Schematic Design.} Draw the circuit using the Block Editor.
    \item Compile the schematic design and perform a functional simulation to verify its correctness.
    \item \textbf{Next, implement the circuit using Structural VHDL Design.}
    \item Write a VHDL file that provides the necessary functionality. Use KEY0 as an active-low asynchronous reset, and use KEY1 as a clock input.
    \item Include the necessary pin assignments for the pushbutton switches and 7-segment displays, and then compile the circuit.
    \item Download the circuit onto your DE-series board and test its functionality by toggling the switches and observing the output displays.
\end{enumerate}

\end{document}

\begin{circuitikz}
    \ctikzset{logic ports=ieee}
    \ctikzset{flipflops/scale=1}
    % Define standard D Latch style
    \tikzset{d-latch/.style={flipflop, flipflop def={
        t1=D, t3=C, t4={\ctikztextnot{Q}}, t6=Q
    }}}
    % Define Negated Enable D Latch style (for Slave)
    \tikzset{d-latch-n/.style={flipflop, flipflop def={
        t1=D, t3=C, t4={\ctikztextnot{Q}}, t6=Q
    }}}

    % Master Latch (Active High Enable)
    \node[d-latch] (Master) at (0,0) {Master};
    
    % Slave Latch (Active Low Enable via Bubble)
    \node[d-latch-n] (Slave) at (5,0) {Slave};
    
    % Inputs
    % Input D
    \draw (Master.pin 1) -- ++(-1.5, 0) coordinate (D_in) node[left] {D};
    
    % Clock Input common point
    % Align Clock horizontal start point with D node x-coordinate
    \coordinate (Clk_pin) at (Master.pin 3);
    
    % Clock Source
    \coordinate (Clk_in_Node) at ($(D_in) + (0,-2.5)$); % Lower down
    \draw (Clk_in_Node) node[left] {Clk} -- ++(1.5,0) coordinate (clk_main_split);
    \fill (clk_main_split) circle (2pt);
    
    % Connect Clock to Master Enable (Active High)
    % Route: Split -> Right -> Up -> Master Pin 3
    \draw (clk_main_split) -| (Master.pin 3);
    
    % Connect Clock to Slave Enable (Active Low)
    % Route: Split -> Right -> Right -> Up -> Slave Pin 3
    \draw (clk_main_split) -- ++(3,0) node[not port,anchor=in,scale=0.5] (not_port) {};
    \draw (not_port.out) -| (Slave.pin 3);
    
    % Logic:
    % Clk=1: Master En (Transp), Slave Dis (Hold).
    % Clk=0: Master Dis (Hold), Slave En (Transp).
    % Negative Edge Triggered.

    % Wiring Master Q -> Slave D
    \draw (Master.pin 6) -- (Slave.pin 1);
    
    % Outputs
    \draw (Slave.pin 6) -- ++(1, 0) node[right] {$Q$};
    \draw (Slave.pin 4) -- ++(1, 0) node[right] {$\bar{Q}$};

\end{circuitikz}
